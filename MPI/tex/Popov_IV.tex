\documentclass[a4paper,12pt]{article}
\usepackage[utf8]{inputenc}
\usepackage[english,russian]{babel}
\usepackage[T2A]{fontenc}

\usepackage[
  a4paper, mag=1000, includefoot,
  left=1.1cm, right=1.1cm, top=1.2cm, bottom=1.2cm, headsep=0.8cm, footskip=0.8cm
]{geometry}

\usepackage{amsmath}
\usepackage{amssymb}
\usepackage{times}
\usepackage{mathptmx}

\IfFileExists{pscyr.sty}
{
  \usepackage{pscyr}
  \def\rmdefault{ftm}
  \def\sfdefault{ftx}
  \def\ttdefault{fer}
  \DeclareMathAlphabet{\mathbf}{OT1}{ftm}{bx}{it} % bx/it or bx/m
}

\mathsurround=0.1em
\clubpenalty=1000%
\widowpenalty=1000%
\brokenpenalty=2000%
\frenchspacing%
\tolerance=2500%
\hbadness=1500%
\vbadness=1500%
\doublehyphendemerits=50000%
\finalhyphendemerits=25000%
\adjdemerits=50000%


\begin{document}

\author{Попов Илья}


\title{Метод Жордана нахождения обратной матрицы с выбором главного элемента по строке}
\date{\today}
\maketitle

\section{Блочный вариант нахождения обратной матрицы методом Жордана с поиском 
главного элемента по строке, параллельный MPI}
\begin{center}
{Постановка задачи}
\end{center}

Находим матрицу обратную к данной
$$A=
   \begin{pmatrix}
     a_{11}& a_{12} &\ldots & a_{1n}\\
     a_{21}& a_{22} &\ldots & a_{2n}\\
     \vdots& \vdots &\ddots & \vdots\\
     a_{n1}& a_{n2} &\ldots & a_{nn}
    \end{pmatrix}
$$
Пусть m - размер блока, тогда поделим n - размер матрицы на m с остатком $n = m*k + l$ тогда матрицу
можно представить в виде:
$$A=
  \begin{pmatrix} 
    A_{11}^{m \times m} & A_{12}^{m \times m} & \cdots & A_{1,k}^{m \times m} & A_{1,k+1}^{m \times l} \\
    A_{21}^{m \times m} & A_{22}^{m \times m} & \cdots & A_{2,k}^{m \times m} & A_{2,k+1}^{m \times l} \\ 
    \vdots & \vdots & \ddots & \vdots & \vdots \\ 
    A_{k,1}^{m \times m} & A_{k,2}^{m \times m} & \cdots & A_{k,k}^{m \times m} & A_{k,k+1}^{m \times l} \\
    A_{k+1,1}^{l \times m} & A_{k+1,2}^{l \times m} & \cdots & A_{k+1,k}^{l \times m} & A_{k+1,k+1}^{l \times l} 
  \end{pmatrix}
$$

\subsection{Разеделение данных на свои и чужие}
Каждый процесс получает блочные строки матрицы с номерами $num + p * counter$, где $num$ это номер потока, 
$p$ это общее число потоков и $num + p * counter <= (k + 1)$, $counter$ это натуральные числа и 0\\
В присоеденненной матрице разделение на потоки происходит в точности таким же образом\\
Каждый процесс имеет доступ к своим данным и не имеет доступа к чужим\\
\subsubsection{Локальная и глобальная нумерации}
j\_glob = j\_loc\\
int local\_to\_global(int m, int p, int k, int i\_loc)\{ \\
      int i\_loc\_m = i\_loc/m;\\
      int i\_glob\_m = i\_loc\_m * p + k;\\
      return i\_glob\_m * m + i\_loc\%m;\\
\}\\
int global\_to\_local(int m, int p, int i\_glob)\{\\
  int i\_glob\_m = i\_glob/m;\\
  int i\_loc\_m = i\_glob\_m/p;\\
  k = i\_glob\_m\%p;\\
  return i\_loc\_m * m + i\_glob\%m;\\
\}\\
Хоть хранение матрицы и блочное но эти формулы нужны в случае заполнения матрицы по формуле и в этом случае они верны\\


\subsection{Формулы в локальной нумерации}
Шаг с номером $h$, где $h <= k$: (то есть $h$ это номера блочных строк в глобальной нумерации)\\
\begin{enumerate}
\item Если блочная строчка с номером $h$ приадлежит потоку с номером $num$ ($num = h\%p$)
    то этот поток рассылает часть это блочной строчки остальным функцией $MPI\_Scatter$, а именно :
    если $begin$ это указатель на начало блочной строчки, то в качестве $sendbuf$ в функцию подается
    $begin + h * m * m$, размер отпарвляемых данных : $(k + 1 - h + p - 1)/p$ таким образом каждый процесс получит часть текущей блочной строки.
    В присоеденненной матрице рассылается между процессами первые $(h/p) * p$ блоков той же функцией
    $MPI\_Scatter$, размер отпрваляемых данных  $(h + 1 + p - 1)/p$
\item Каждый процесс получил $q$ блоков исходной матрицы в свой буффер и среди них ищет блок с наименьшей нормой обратного, а именно :\\
    среди блоков $Buf1_{0, g}, g = p, p <= q$ \\
\item С момощю функции $MPI\_Allreduce$ потоки находят блок с наименьшей нормой обратного среди всех блоков
    данной строки - блок с номером $w$ (номера столбцов совпадают в локальной и глобальной нумерации)
    если такого блока нет, то алгоритм не применим\\
\item Каждый поток в своих строчках меняет местами $k$ и $w$ столбцы\\
\item $MPI\_Bcast$ блока с наименьшей нормой всем остальным от потока который его нашел - блок $S$. 
    Размер отправляемых данных - $m * m$\\
\item Каждй процесс домножает свои блоки в буффере на $S^{-1}$ слева\\
    $Buf1_{0, g}, g = p, p <= q$\\
    $Buf2_{0, g}, g = p, p <= (h + 1 + p - 1)/p)$\\
    , где Buf2 - буффер от присоеденненной матрицы\\
\item Функцией $MPI\_Allgather$ все процессы собирают блочную строчку исходной матрицы и меняют местами блоки $w$ и $h$.
    Размер отпрваляемых данных - блочная строка из $k - h + 1$ блоков.
    Функцией $MPI\_Allgather$ все процессы собирают блочную строчку присоеденнёной матрице
    Размер отправляемых данных - блочная строка из $h + 1$ блоков.
\item Каждый процесс $num$ вычитает из своих строчек строчку из буффера в исходной и присоеденнёной матрице\\
    домноженную слева на ведущий блок этой строки:
        $$A_{i, j} = A_{i, j} - Buf1_{0, j} * A_{i, h}, \quad i < rows, i>=0 , (i != h/p, если num==h\%p)
        \quad j = h + 1,..,k+1$$
        $$B_{i, j} = B_{i, j} - A_{i, h} * Buf2_{0, j}, \quad i < rows, i>= 0, (i != h/p, если num==h\%p) 
        \quad j = 1,...,h$$
    , где $rows$ - это число блочных строчек на данный поток\\

\end{enumerate}
На шаге с номером $k + 1$ ищем обратный блок к блоку $A_{k+1,k+1}$(глобальной) -> $A_{rows - 1, rows - 1}$(локально) = $S$
в процессе с номером $num = (k + 1)\%p$, где $rows$ - число блочных строк в процессе $num$\\
Если блок не обратим то алгоритм не применим для данного $m$.\\
$MPI\_Bcast$ $S^{-1}$ всем остальным процессам\\
Проводим выше описанный алгоритм только для присоеденненной матрицы и только для последней блочной строчки\\\\

Так как в матрице $A$ мы меняли мастами стобцы для нахождения главного элемента, то 
в матрице присоеденённой $B$ мы меняем мастими соответствующие строки.

\subsection{Точки коммуникации}
Были явно описаны в ходе описания алгоритма

\subsection{Формула сложности}
\subsubsection{Число обменов}
$$C(n, m, p) = 6 * \frac{n}{m}$$
\subsubsection{Объем обменов}
$$V = m^{2} * \frac{n}{m} + 2*\frac{1}{p} * \frac{n}{m} * \frac{p*m*(p*m - 1)}{2} +$$
$$2*\sum\limits_{i=1}^{\frac{n}{p * m}}p * i * m^{2}$$
$$+ n^{2} = n*m + n*m*p - n + m*n + \frac{n^{2}}{p} = 2*n*m + n*m*p -n + \frac{n^{2}}{p} + n^{2}$$

$$V(n, m, p) = n^{2}*(1 + \frac{1}{p}) + 2*n*m + n*m*p - n$$

\subsubsection{Сложность}
Для рассчета формулы сложности считаем, что $n \% m == 0, \frac{n}{m} \% p == 0$\\
$n = q * m$, $p$ - число потоков\\
\subsubsection{Для исходной матрицы}
$$\sum\limits_{k=1}^{q}\lceil \frac{k}{p} \rceil * (2 * m^{3} - \frac{m^{2}}{2} - \frac{m}{2}) +
\sum\limits_{k=1}^{q-1}\lceil \frac{q-k}{p} \rceil*(2*m^{3}-m^{2}) +
\sum\limits_{k=1}^{q-1} (q-k)*2*m^{3} * (\lceil \frac{q-1}{p} \rceil)$$
Так как $\sum\limits_{i=1}^{q}\lceil \frac{i}{p} \rceil =
\sum\limits_{i=1}^{p}1 + \sum\limits_{i=p+1}^{2*p}2 + ... = p * \sum\limits_{i=1}^{\frac{n}{m*p}}
1 = \frac{q^{2}}{2*p} + \frac{q}{2}$\\
То получаем, что 
$$(2*m^{3} - \frac{m^{2}}{2} - \frac{m}{2})*(\frac{q^{2}}{2*p} + \frac{q}{2}) + 
(2*m^{3} - m^{2}) *(\frac{q^{2}}{2*p} - \frac{q}{2}) + m^{3}*(q-1)^{2}*\frac{q}{p} = $$
$$ = \frac{n^{3}}{p} + \frac{n*m^{2}}{p} + O(n^{2} + n*m + m^{2})$$
\subsubsection{Для присоеденнёной}
$$\sum\limits_{k=1}^{q}(\lceil \frac{k-1}{p} \rceil)* (2*m^{3}-m^{2}) + \lceil \frac{q-1}{p}
\rceil * \sum\limits_{k=1}^{q}k*2*m^{3} = $$
$$= (2*m^{3} - m^{2}) * (\frac{q^{2}}{2*p}-\frac{q}{2}) + 2*m^{3}*(q^{2}-1)*\frac{q}{p} =
\frac{n^{3}}{p} + \frac{n^{2}*m}{p} - 2*\frac{n*m{2}}{p} + O(n^{2} + n*m + m^{2})$$
Суммируя получаем итогувую сложность :\\
$$S_{p}(n, m , p) = 2 * \frac{n^{3}}{p} + \frac{n^{2}*m}{p} - \frac{n*m^{2}}{p} +
O(n^{2} + n*m + m^{2})$$
$$S_{p}(n,m,1) = S(n, m)$$

\end{document}










