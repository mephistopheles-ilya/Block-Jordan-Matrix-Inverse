\documentclass[a4paper,12pt]{article}
\usepackage[utf8]{inputenc}
\usepackage[english,russian]{babel}
\usepackage[T2A]{fontenc}

\usepackage[
  a4paper, mag=1000, includefoot,
  left=1.1cm, right=1.1cm, top=1.2cm, bottom=1.2cm, headsep=0.8cm, footskip=0.8cm
]{geometry}

\usepackage{amsmath}
\usepackage{amssymb}
\usepackage{times}
\usepackage{mathptmx}

\IfFileExists{pscyr.sty}
{
  \usepackage{pscyr}
  \def\rmdefault{ftm}
  \def\sfdefault{ftx}
  \def\ttdefault{fer}
  \DeclareMathAlphabet{\mathbf}{OT1}{ftm}{bx}{it} % bx/it or bx/m
}

\mathsurround=0.1em
\clubpenalty=1000%
\widowpenalty=1000%
\brokenpenalty=2000%
\frenchspacing%
\tolerance=2500%
\hbadness=1500%
\vbadness=1500%
\doublehyphendemerits=50000%
\finalhyphendemerits=25000%
\adjdemerits=50000%


\begin{document}

\author{Попов Илья}


\title{Метод Жордана нахождения обратной матрицы с выбором главного элемента по строке}
\date{\today}
\maketitle

\section{Блочный вариант нахождения обратной матрицы методом Жордана с поиском 
главного элемента по строке, параллельный}
\begin{center}
{Постановка задачи}
\end{center}

Находим матрицу обратную к данной
$$A=
   \begin{pmatrix}
     a_{11}& a_{12} &\ldots & a_{1n}\\
     a_{21}& a_{22} &\ldots & a_{2n}\\
     \vdots& \vdots &\ddots & \vdots\\
     a_{n1}& a_{n2} &\ldots & a_{nn}
    \end{pmatrix}
$$
Пусть m - размер блока, тогда поделим n - размер матрицы на m с остатком $n = m*k + l$ тогда матрицу
можно представить в виде:
$$A=
  \begin{pmatrix} 
    A_{11}^{m \times m} & A_{12}^{m \times m} & \cdots & A_{1,k}^{m \times m} & A_{1,k+1}^{m \times l} \\
    A_{21}^{m \times m} & A_{22}^{m \times m} & \cdots & A_{2,k}^{m \times m} & A_{2,k+1}^{m \times l} \\ 
    \vdots & \vdots & \ddots & \vdots & \vdots \\ 
    A_{k,1}^{m \times m} & A_{k,2}^{m \times m} & \cdots & A_{k,k}^{m \times m} & A_{k,k+1}^{m \times l} \\
    A_{k+1,1}^{l \times m} & A_{k+1,2}^{l \times m} & \cdots & A_{k+1,k}^{l \times m} & A_{k+1,k+1}^{l \times l} 
  \end{pmatrix}
$$

\subsection{Разеделение данных на свои и чужие}
Каждый поток получает строки матрицы с номерами $num + p * counter$, где $num$ это номер потока, 
$p$ это общее число потоков и $num + p * counter <= (k + 1)$, $counter$ это натуральные числа и 0\\
В присоеденненной матрице разделение на потоки происходит в точности таким же образом

\subsection{Формулы на поток}
Шаг с номером $h$, где $h <= k$:\\
\begin{enumerate}
\item Поток с номером $num$ находит обратные матрицы среди матриц строки $h$, а именно
среди матриц $A_{h,g}, g = num + p * counter <= k, g >= h$. Для обратных блоков считаются нормы и
и выбирается блок с наименьшей нормой обратного\\
\item Первая точка синхронизации ( $reduce\_sum$ ): потоки находят блоки с наименьшей нормой среди
наименьших, но ответ уже для всей строки. Пусть наименьший блок $A_{h, j}$, если во всех потоках 
не удалось найти обратный, то данный алгоритм не применим\\
\item Каждый поток в своих строчках меняет местими столбцы с номерами $h$ и $j$\\
\item Вторая точка синхронизации ( $barrier$ )\\
\item Все потоки находят обратнцю матрицу к блоку $A_{h, h}$\\
\item Поток с номером $num$ домножает слева на $A_{h, h}^{-1}$ блоки 
    $A_{h, j}, j = num + p * counter <= (k + 1)
        , j >= h$ и $B_{h, i}, i = num + p * counter <= h, i >= 1$
,где блоки $B$ это блоки присоеденненной матрицы\\
\item Третья точка синхронизации ( $barrier$ )\\
\item Поток с номером $num$:\\
В матрицах $A$ и $B$ вычитаем из строк с номером $i != h$ строку с номером $h$ 
        домноженную слева на $A_{i, h}$ 
        А именно: 
        $$A_{i, j} = A_{i, j} - A_{i, q} * A_{q, j}, \quad i = num + p * counter <= (k + 1), i>= 1,
        i != h
        \quad j = q + 1,..,k+1$$
        $$B_{i, j} = B_{i, j} - A_{i, q} * B_{q, j}, \quad i = num + p * counter <= (k + 1), i>= 1,
        i != h
        \quad j = 1,...,h$$
\item Четвертая точка синхронизации ( $barrier$ ) перед следеющим шагом алгоритма
\end{enumerate}
На шаге с номером $k + 1$ ищем обратный блок к блоку $A_{k+1,k+1}$ вовсех потоках,
если этот блок не обратим,
то алгоритм не применим для данного $m$. В потоке с номером  
$num$ только в матрице $B$ выполняем следующие:
$$B_{h, i}, i = num + p * counter <= (k + 1), i >= 1$$
Точка синхронизации ($barrier$)
$$B_{i,j} = B_{i,j} - A_{i,q} * B_{q,j}, \quad i = num + p * counter <= (k + 1)
, i >= 1, i != k + 1, \quad j = 1,...,k+1$$
Точка синхронизации ( $barrier$ )

Так как в матрице $A$ мы меняли мастами стобцы для нахождения главного элемента, то 
в матрице присоеденённой $B$ мы меняем мастими соответствующие строки.


\subsection{Формула сложности}
Для рассчета формулы сложности считаем, что $n \% m == 0, \frac{n}{m} \% p == 0$\\
$n = q * m$, $p$ - число потоков\\
\subsubsection{Для исходной матрицы}
$$\sum\limits_{k=1}^{q}\lceil \frac{k}{p} \rceil * (2 * m^{3} - \frac{m^{2}}{2} - \frac{m}{2}) +
\sum\limits_{k=1}^{q-1}\lceil \frac{q-k}{p} \rceil*(2*m^{3}-m^{2}) +
\sum\limits_{k=1}^{q-1} (q-k)*2*m^{3} * (\lceil \frac{q-1}{p} \rceil)$$
Так как $\sum\limits_{i=1}^{q}\lceil \frac{i}{p} \rceil =
\sum\limits_{i=1}^{p}1 + \sum\limits_{i=p+1}^{2*p}2 + ... = p * \sum\limits_{i=1}^{\frac{n}{m*p}}
1 = \frac{q^{2}}{2*p} + \frac{q}{2}$\\
То получаем, что 
$$(2*m^{3} - \frac{m^{2}}{2} - \frac{m}{2})*(\frac{q^{2}}{2*p} + \frac{q}{2}) + 
(2*m^{3} - m^{2}) *(\frac{q^{2}}{2*p} - \frac{q}{2}) + m^{3}*(q-1)^{2}*\frac{q}{p} = $$
$$ = \frac{n^{3}}{p} + \frac{n*m^{2}}{p} + O(n^{2} + n*m + m^{2})$$
\subsubsection{Для присоеденнёной}
$$\sum\limits_{k=1}^{q}(\lceil \frac{k-1}{p} \rceil)* (2*m^{3}-m^{2}) + \lceil \frac{q-1}{p}
\rceil * \sum\limits_{k=1}^{q}k*2*m^{3} = $$
$$= (2*m^{3} - m^{2}) * (\frac{q^{2}}{2*p}-\frac{q}{2}) + 2*m^{3}*(q^{2}-1)*\frac{q}{p} =
\frac{n^{3}}{p} + \frac{n^{2}*m}{p} - 2*\frac{n*m{2}}{p} + O(n^{2} + n*m + m^{2})$$
Суммируя получаем итогувую сложность :\\
$$S_{p}(n, m , p) = 2 * \frac{n^{3}}{p} + \frac{n^{2}*m}{p} - \frac{n*m^{2}}{p} +
O(n^{2} + n*m + m^{2})$$
$$S_{p}(n,m,1) = S(n, m)$$




\subsection{Оценка числа точек синхронизации}
$$4 * (\frac{n}{m}) + 2$$

\section{Метод Жордана нахождения обратной матрицы с выбором главного элемента по строке}
\begin{center}
{Постановка задачи}
\end{center}

Находим матрицу обратную к данной
$$A=
   \begin{pmatrix}
     a_{11}& a_{12} &\ldots & a_{1n}\\
     a_{21}& a_{22} &\ldots & a_{2n}\\
     \vdots& \vdots &\ddots & \vdots\\
     a_{n1}& a_{n2} &\ldots & a_{nn}
    \end{pmatrix}
$$
Пусть m - размер блока, тогда поделим n - размер матрицы на m с остатком $n = m*k + l$ тогда матрицу
можно представить в виде:
$$A=
  \begin{pmatrix} 
    A_{11}^{m \times m} & A_{12}^{m \times m} & \cdots & A_{1,k}^{m \times m} & A_{1,k+1}^{m \times l} \\
    A_{21}^{m \times m} & A_{22}^{m \times m} & \cdots & A_{2,k}^{m \times m} & A_{2,k+1}^{m \times l} \\ 
    \vdots & \vdots & \ddots & \vdots & \vdots \\ 
    A_{k,1}^{m \times m} & A_{k,2}^{m \times m} & \cdots & A_{k,k}^{m \times m} & A_{k,k+1}^{m \times l} \\
    A_{k+1,1}^{l \times m} & A_{k+1,2}^{l \times m} & \cdots & A_{k+1,k}^{l \times m} & A_{k+1,k+1}^{l \times l} 
  \end{pmatrix}
$$

\subsection{Хранение матрицы в памяти}
Матрицу будем хранить в памяти следующим образом
$$
A = \left(
\begin{array}{c c c c | c c c c | c | c c c | c c c c | c}
    a_{11} & a_{12} & \dots & a_{1,m} & a_{21} & a_{22} & \dots & a_{2,m} & \dots & a_{m,1} &
    \dots & a_{m,m} & a_{1,m+1} & a_{1,m+2} & \dots & a_{1,m+m} & \dots
\end{array}
\right)
$$
То есть хранение блочное
$$A=
  \begin{pmatrix} 
      A_{11}^{m \times m} & A_{12}^{m \times m} & \cdots & A_{1,k}^{m \times m} &
      A_{1,k+1}^{m \times l} & \cdots & A_{k+1,1}^{l \times m} & \cdots & 
      A_{k+1,k}^{l \times m} & A_{k+1,k+1}^{l \times l} 
  \end{pmatrix}
$$

\subsection{Описание функций getBlock и setBlock}
Функции getBlock и setBlock возвращают указатель на начало блока. Адресс блока с номером
$(i, j)$ это $$A + (i - 1) * (m * m * k + m * l) + (j - 1) * m * m$$ где $A$ - это адресс 
начала матрицы.
Если же $i = k + 1$, то адресс будет равняться 
$$A + k * (m * m * k + m * l) + (j - 1) * l * m$$

\subsection{Описание формул}
Пусть $B$ - присоеденённая матрица, которая тоже хранится блоками.\\
Обратная матрица для блоков находится с помощью обычного метода Жордана c выбором главного
элемента по строке.\\
Норма матрицы : 
$\|A^{m \times m} \| = \max\limits_{i = 1,\ldots,m} \sum\limits_{j = 1}^m |a_{ij}|$\\
$q$ - шаг меняется от $1$ до $k$ включительно

\begin{enumerate}
    \item На $q$-ом шаге алгоритма среди блоков $A_{q,j}$ при $j = q,...,k$ выбираем блок,
        у которой норма обратного наименьшая. Для этого данный блок обращается и у него
        считается норма, если блок не обратим, то пропускаем его.
        Если все блоки среди $A_{q,j}$ при $j = q,...,k$ необратимы то данный алгоритм не применим.
        Пусть у блока $A_{q, g}$ норма обратной наименшая, тога меняем местами столбцы
        (блочные) с номерами 
        $q$ и $g$. Матрицу $B$ на данном шаге не трогаем. Запоминаем номерая столбцов которые
        поменяли местами.
    \item Умножаем все блоки $A_{q, j}$ при $j = q + 1,...,k, k + 1$ слева на к $A_{q, q}^{-1}$.
        Блок $A_{q,q}$ на $A_{q, q}^{-1}$ не домножаем, просто записываем еденичный блок
        (этого можно не делать так как обращений к этому блоку больше не будет)
        В матрице $B$ домножаем слева на $A_{q, q}^{-1}$ блоки $B_{q, j}$ при 
        $j = 1,..,q-1$.
        На место блока $B_{q,q}$ сразу запишем $A_{q,q}^{-1}$ так ка изначално он еденичный.
        $$A_{q, j} = A_{q, q}^{-1} * A_{q, j}, \quad j = q + 1,...,k, k + 1$$ 
        $$B_{q, j} = A_{q, q}^{-1} * B_{q, j}, \quad j = 1,...,q-1$$ 
    \item В матрицах $A$ и $B$ вычитаем из строк с номером $i != q$ строку с номером $q$ 
        домноженную слева на $A_{i, q}$ 
        А именно: 
        $$A_{i, j} = A_{i, j} - A_{i, q} * A_{q, j}, \quad i = 1,...,q-1,q+1,...,k+1,
        \quad j = q + 1,..,k+1$$
        $$B_{i, j} = B_{i, j} - A_{i, q} * B_{q, j}, \quad i = 1,...,q-1,q+1,...,k+1,
        \quad j = 1,...,q$$
\end{enumerate}
На шаге с номером $k + 1$ ищем обратный блок к блоку $A_{k+1,k+1}$ если этот блок не обратим,
то алгоритм не применим для данного $m$. Только в матрице $B$ выполняем следующие:
$$B_{k+1,k+1} = A_{k+1,k+1}^{-1}$$
$$B_{i,j} = B_{i,j} - A_{i,q} * B_{q,j}, \quad i = 1,...,k, \quad j = 1,...,k+1$$

Так как в матрице $A$ мы меняли мастами стобцы для нахождения главного элемента, то 
в матрице присоеденённой $B$ мы меняем мастими соответствующие строки.

\subsection{Оценка числа операций (вариант 1)}
Умножение двух матриц $A^{n \times m} * A^{m \times k} : n*k(m + m - 1) = 2*n*k*m - m*k$\\
Сложение или вычитание двух матриц $A^{n \times m} + A^{n \times m}: n*m$\\
Нахождение обратной матрицы $A^{m \times m}$ обычным методом 
Жордана с посиком главного элемена по строке $2*m^{3} + \frac{11}{2}*m^{2} - \frac{13}{2}m$\\
Расчеты для не блочного метода:\\
Для исходной матрицы:
$$\sum\limits_{k = 1}^{m-1}(m - k) + 2 *(m - 1)*\sum\limits_{k = 1}^{m-1}(m-k) + 3*m^{2} - 3*m =
m^{2} - m + m*(m^{2} - 2*m + 1) + 3*m^{2} - 3*m = m^{3} + 2*m - 3*m$$
Для присоеденённой:
$$\sum\limits_{k=1}^{m}k  + 2 * (m -1)*\sum\limits_{k=1}^{m}k + 3*m^{2} - 3*m = m^{3} + \frac{7}{2}
*m^{2} - \frac{7}{2}*m$$
Складывая получаем:
$$2*m^{3} + \frac{11}{2} * m^{2} - \frac{13}{2}*m$$
Для блочного варианта: (пусть $q = n/m$ и не будем учитывать не поделившиеся края, так как
они не влияют на ассимптотику)\\
Для присоеденённой матрицы:
$$\sum\limits_{k=1}^{q}(2*m^{3} - m^{2}) + \sum\limits_{k=1}^{q}(q-1)*k*(2*m^{3} - m^{2})
+ \sum\limits_{k=1}^{q}(q-1)*k*m^{2} =$$
$$= (2*m^{3} - m^{2}) * (\frac{q^{2}}{2} + \frac{q}{2})
+ (2 * m^{3} - m^{2})*(q^{2} - 1)*\frac{1}{2} + m^{2} * (q^{2} - 1) * \frac{q}{2} = 
n^{3} + n^{2}*m + O(n^{2} + n*m + m^{2})$$
Для исходной матрицы:
$$(2*m^{3} + \frac{11}{2}*m^{2} - \frac{13}{2}*m) * \frac{1 + q}{2} * q + 
\sum\limits_{k=1}^{q-1}(q - k)*(2*m^{3} - m^{2})*(q -1) + 
\sum\limits_{k=1}^{q-1}(q-1)(q-k)*m^{2} = $$
$$ = (2*m^{3} + \frac{11}{2}*m^{2} - \frac{13}{2}*m) * (\frac{q^{2}}{2} + \frac{q}{2}) + 
(2*m^{3} - m^{2})*(\frac{q^{2}}{2} - \frac{q}{2}) + 2*m^{3} * (q - 1)^{2} * \frac{1}{2} = $$
$$ = n^{3} + m^{2}*n + O(n^{2} + n*m + m^{2})$$
Складывая, получаем:
$$2*n^{3} + m^{2} * n + n^{2} * {m} +O (n^{2} + n*m + m^{2})$$
\subsection{Проверка формулы (вариант 1)}
$S(n, 1) = 2*n^{3} + O(n^{2})$\\
$S(n, n) = 4*n^{3} + O(n^{2})$

\subsection{Оценка числа операций (Вариант 2)}
$q = \frac{n}{m}$\\
\subsubsection{Для исходной матрицы:}
Cложность нахождения обратных блоков для поиска ведущего среди $A_{k,j}$ при $j = k,...,q$ :\\
$$(2*m^{3} - \frac{m^{2}}{2} - \frac{m}{2}) * \frac{1 + q}{2} * q$$ 
Сложность умножений на блок, обратный к ведущему $A_{k, j} = A_{k, k}^{-1} * A_{k, j},
\quad j = k + 1,...,q$ :\\
$$\sum\limits_{k=1}^{q-1}(q - k)*(2*m^{3}-m^{2})$$
Сложность умножений и вычитаний для 
$A_{i, j} = A_{i, j} - A_{i, k} * A_{k, j}, \quad i = 1,...,k-1,k+1,...,q,
\quad j = k + 1,..,q$:\\
$$\sum\limits_{k=1}^{q-1}(q-k)*2*m^{3}*(q-1)$$
Тогда, суммируя, получим: 
$$ 2*m^{3}*(\frac{q^{2}}{2} + \frac{q}{2}) + (2*m^{3}-m^{2})*(\frac{q^{2}}{2} - \frac{q}{2})
+ m^{3}(q - 1)^{2}*q = $$
$$=\frac{1}{2} * (2*m^{3}) * (\frac{n^{2}}{m^{2}} + \frac{n}{m}) + \frac{1}{2} * (2*m^{3} - m^{2})
*(\frac{n^{2}}{m^{2}} - \frac{n}{m}) + m^{3}*(\frac{n^{3}}{m^{3}} - 2*\frac{n^{2}}{m^{2}} +
\frac{n}{m}) = $$
$$ = n^{3} + n*m^{2} + O(n^{2} + m*m + m^{2})$$
\subsubsection{Для присоеденнёной матрицы:}
Сложность умножений на блок, обратный к ведущему $B_{k, j} = A_{k, k}^{-1} * B_{k, j},
\quad j = 1,...,k-1$ :\\
$$\sum\limits_{k=1}^{q}(k-1)*(2*m^{3} - m^{2})$$
Сложность умножений и вычитаний для 
$B_{i, j} = B_{i, j} - A_{i, k} * B_{k, j}, \quad i = 1,...,k-1,k+1,...,q,
\quad j = 1,...,k$:\\
$$(q-1)*\sum\limits_{k=1}^{q}*k*2*m^{3}$$
\subsubsection{Тогда, суммируя, получим:}
$$(2*m^{3} - m^{2})*(\frac{q^{2}}{q} - \frac{q}{2}) + m^{3}*(q^{3} - q) = 
\frac{1}{2}*(2*n^{2}*m - 2*n*m{2}) = n^{3} - n*m^{2} = $$
$$ = n^{2}*m + n^{3} - m^{2} * n + O(n^{2} + m*m + m^{2})$$
\subsubsection{Итог:}
$$S(n, m) = 2*n^{3} + n^{2}*m - n*m^{2} + O(n^{2} + m*m + m^{2})$$

\subsection{Проверка формулы (вариант 2)}
$S(n, 1) = 2*n^{3} + O(n^{2})$ - сложность обычного метода Жордана нахождений обратной матрицы\\
$S(n, n) = 2*n^{3} + O(n^{2})$ - сложность обычного метода Жордана нахождения обратной матрицы,
но нет ешё одного умножения, так как в присоеденённой матрице еденичный блок на 
$B_{k, k}$ на шаге $k$ на $A_{k,k}^{-1}$ не умножается, а просто заменяется
(если все таки умножать, то получится вариант 1, то есть при $m=n$ будет
сложность = Жордан + одно умножений = $4*n^{3})$

\end{document}










