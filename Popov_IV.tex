\documentclass[a4paper,12pt]{article}
\usepackage[utf8]{inputenc}
\usepackage[english,russian]{babel}
\usepackage[T2A]{fontenc}

\usepackage[
  a4paper, mag=1000, includefoot,
  left=1.1cm, right=1.1cm, top=1.2cm, bottom=1.2cm, headsep=0.8cm, footskip=0.8cm
]{geometry}

\usepackage{amsmath}
\usepackage{amssymb}
\usepackage{times}
\usepackage{mathptmx}

\IfFileExists{pscyr.sty}
{
  \usepackage{pscyr}
  \def\rmdefault{ftm}
  \def\sfdefault{ftx}
  \def\ttdefault{fer}
  \DeclareMathAlphabet{\mathbf}{OT1}{ftm}{bx}{it} % bx/it or bx/m
}

\mathsurround=0.1em
\clubpenalty=1000%
\widowpenalty=1000%
\brokenpenalty=2000%
\frenchspacing%
\tolerance=2500%
\hbadness=1500%
\vbadness=1500%
\doublehyphendemerits=50000%
\finalhyphendemerits=25000%
\adjdemerits=50000%


\begin{document}

\author{Попов Илья}
\title{Метод Жордана нахождения обратной матрицы с выбором главного элемента по строке}
\date{\today}
\maketitle

\begin{center}
{Постановка задачи}
\end{center}

Находим матрицу обратную к данной
$$A=
   \begin{pmatrix}
     a_{11}& a_{12} &\ldots & a_{1n}\\
     a_{21}& a_{22} &\ldots & a_{2n}\\
     \vdots& \vdots &\ddots & \vdots\\
     a_{n1}& a_{n2} &\ldots & a_{nn}
    \end{pmatrix}
$$
Пусть m - размер блока, тогда поделим n - размер матрицы на m с остатком $n = m*k + l$ тогда матрицу
можно представить в виде:
$$A=
  \begin{pmatrix} 
    A_{11}^{m \times m} & A_{12}^{m \times m} & \cdots & A_{1,k}^{m \times m} & A_{1,k+1}^{m \times l} \\
    A_{21}^{m \times m} & A_{22}^{m \times m} & \cdots & A_{2,k}^{m \times m} & A_{2,k+1}^{m \times l} \\ 
    \vdots & \vdots & \ddots & \vdots & \vdots \\ 
    A_{k,1}^{m \times m} & A_{k,2}^{m \times m} & \cdots & A_{k,k}^{m \times m} & A_{k,k+1}^{m \times l} \\
    A_{k+1,1}^{l \times m} & A_{k+1,2}^{l \times m} & \cdots & A_{k+1,k}^{l \times m} & A_{k+1,k+1}^{l \times l} 
  \end{pmatrix}
$$

\section{Хранение матрицы в памяти}
Матрицу будем хранить в памяти следующим образом
$$
A = \left(
\begin{array}{c c c c | c c c c | c | c c c | c c c c | c}
    a_{11} & a_{12} & \dots & a_{1,m} & a_{21} & a_{22} & \dots & a_{2,m} & \dots & a_{m,1} &
    \dots & a_{m,m} & a_{1,m+1} & a_{1,m+2} & \dots & a_{1,m+m} & \dots
\end{array}
\right)
$$
То есть хранение блочное
$$A=
  \begin{pmatrix} 
      A_{11}^{m \times m} & A_{12}^{m \times m} & \cdots & A_{1,k}^{m \times m} &
      A_{1,k+1}^{m \times l} & \cdots & A_{k+1,1}^{l \times m} & \cdots & 
      A_{k+1,k}^{l \times m} & A_{k+1,k+1}^{l \times l} 
  \end{pmatrix}
$$

\section{Описание функций getBlock и setBlock}
Функции getBlock и setBlock возвращают указатель на начало блока. Адресс блока с номером
$(i, j)$ это $$A + (i - 1) * (m * m * k + m * l) + (j - 1) * m * m$$ где $A$ - это адресс 
начала матрицы.
Если же $i = k + 1$, то адресс будет равняться 
$$A + k * (m * m * k + m * l) + (j - 1) * l * m$$

\section{Описание формул}
Пусть $B$ - присоеденённая матрица, которая тоже хранится блоками.\\
Обратная матрица для блоков находится с помощью обычного метода Жордана c выбором главного
элемента по строке.\\
Норма матрицы : 
$\|A^{m \times m} \| = \max\limits_{i = 1,\ldots,m} \sum\limits_{j = 1}^m |a_{ij}|$\\
$q$ - шаг меняется от $1$ до $k$ включительно

\begin{enumerate}
    \item На $q$-ом шаге алгоритма среди блоков $A_{q,j}$ при $j = q,...,k$ выбираем блок,
        у которой норма обратного наименьшая. Для этого данный блок обращается и у него
        считается норма, если блок не обратим, то пропускаем его.
        Если все блоки среди $A_{q,j}$ при $j = q,...,k$ необратимы то данный алгоритм не применим.
        Пусть у блока $A_{q, g}$ норма обратной наименшая, тога меняем местами столбцы
        (блочные) с номерами 
        $q$ и $g$. Матрицу $B$ на данном шаге не трогаем. Запоминаем номерая столбцов которые
        поменяли местами.
    \item Умножаем все блоки $A_{q, j}$ при $j = q + 1,...,k, k + 1$ слева на к $A_{q, q}^{-1}$.
        Блок $A_{q,q}$ на $A_{q, q}^{-1}$ не домножаем, просто записываем еденичный блок
        (этого можно не делать так как обращений к этому блоку больше не будет)
        В матрице $B$ домножаем слева на $A_{q, q}^{-1}$ блоки $B_{q, j}$ при 
        $j = 1,..,q-1$.
        На место блока $B_{q,q}$ сразу запишем $A_{q,q}^{-1}$ так ка изначално он еденичный.
        $$A_{q, j} = A_{q, q}^{-1} * A_{q, j}, \quad j = q + 1,...,k, k + 1$$ 
        $$B_{q, j} = A_{q, q}^{-1} * B_{q, j}, \quad j = 1,...,q-1$$ 
    \item В матрицах $A$ и $B$ вычитаем из строк с номером $i != q$ строку с номером $q$ 
        домноженную слева на $A_{i, q}$ 
        А именно: 
        $$A_{i, j} = A_{i, j} - A_{i, q} * A_{q, j}, \quad i = 1,...,q-1,q+1,...,k+1,
        \quad j = q + 1,..,k+1$$
        $$B_{i, j} = B_{i, j} - A_{i, q} * B_{q, j}, \quad i = 1,...,q-1,q+1,...,k+1,
        \quad j = 1,...,q$$
\end{enumerate}
На шаге с номером $k + 1$ ищем обратный блок к блоку $A_{k+1,k+1}$ если этот блок не обратим,
то алгоритм не применим для данного $m$. Только в матрице $B$ выполняем следующие:
$$B_{k+1,k+1} = A_{k+1,k+1}^{-1}$$
$$B_{i,j} = B_{i,j} - A_{i,q} * B_{q,j}, \quad i = 1,...,k, \quad j = 1,...,k+1$$

Так как в матрице $A$ мы меняли мастами стобцы для нахождения главного элемента, то 
в матрице присоеденённой $B$ мы меняем мастими соответствующие строки.

\section{Оценка числа операций (вариант 1)}
Умножение двух матриц $A^{n \times m} * A^{m \times k} : n*k(m + m - 1) = 2*n*k*m - m*k$\\
Сложение или вычитание двух матриц $A^{n \times m} + A^{n \times m}: n*m$\\
Нахождение обратной матрицы $A^{m \times m}$ обычным методом 
Жордана с посиком главного элемена по строке $2*m^{3} + \frac{11}{2}*m^{2} - \frac{13}{2}m$\\
Расчеты для не блочного метода:\\
Для исходной матрицы:
$$\sum\limits_{k = 1}^{m-1}(m - k) + 2 *(m - 1)*\sum\limits_{k = 1}^{m-1}(m-k) + 3*m^{2} - 3*m =
m^{2} - m + m*(m^{2} - 2*m + 1) + 3*m^{2} - 3*m = m^{3} + 2*m - 3*m$$
Для присоеденённой:
$$\sum\limits_{k=1}^{m}k  + 2 * (m -1)*\sum\limits_{k=1}^{m}k + 3*m^{2} - 3*m = m^{3} + \frac{7}{2}
*m^{2} - \frac{7}{2}*m$$
Складывая получаем:
$$2*m^{3} + \frac{11}{2} * m^{2} - \frac{13}{2}*m$$
Для блочного варианта: (пусть $q = n/m$ и не будем учитывать не поделившиеся края, так как
они не влияют на ассимптотику)\\
Для присоеденённой матрицы:
$$\sum\limits_{k=1}^{q}(2*m^{3} - m^{2}) + \sum\limits_{k=1}^{q}(q-1)*k*(2*m^{3} - m^{2})
+ \sum\limits_{k=1}^{q}(q-1)*k*m^{2} =$$
$$= (2*m^{3} - m^{2}) * (\frac{q^{2}}{2} + \frac{q}{2})
+ (2 * m^{3} - m^{2})*(q^{2} - 1)*\frac{1}{2} + m^{2} * (q^{2} - 1) * \frac{q}{2} = 
n^{3} + n^{2}*m + O(n^{2} + n*m + m^{2})$$
Для исходной матрицы:
$$(2*m^{3} + \frac{11}{2}*m^{2} - \frac{13}{2}*m) * \frac{1 + q}{2} * q + 
\sum\limits_{k=1}^{q-1}(q - k)*(2*m^{3} - m^{2})*(q -1) + 
\sum\limits_{k=1}^{q-1}(q-1)(q-k)*m^{2} = $$
$$ = (2*m^{3} + \frac{11}{2}*m^{2} - \frac{13}{2}*m) * (\frac{q^{2}}{2} + \frac{q}{2}) + 
(2*m^{3} - m^{2})*(\frac{q^{2}}{2} - \frac{q}{2}) + 2*m^{3} * (q - 1)^{2} * \frac{1}{2} = $$
$$ = n^{3} + m^{2}*n + O(n^{2} + n*m + m^{2})$$
Складывая, получаем:
$$2*n^{3} + m^{2} * n + n^{2} * {m} +O (n^{2} + n*m + m^{2})$$
\section{Проверка формулы (вариант 1)}
$S(n, 1) = 2*n^{3} + O(n^{2})$\\
$S(n, n) = 4*n^{3} + O(n^{2})$

\section{Оценка числа операций (Вариант 2)}
$q = \frac{n}{m}$\\
Для исходной матрицы:\\
Cложность нахождения обратных блоков для поиска ведущего среди $A_{k,j}$ при $j = k,...,q$ :\\
$$(2*m^{3} - \frac{m^{2}}{2} - \frac{m}{2}) * \frac{1 + q}{2} * q$$ 
Сложность умножений на блок, обратный к ведущему $A_{k, j} = A_{k, k}^{-1} * A_{k, j},
\quad j = k + 1,...,q$ :\\
$$\sum\limits_{k=1}^{q-1}(q - k)*(2*m^{3}-m^{2})$$
Сложность умножений и вычитаний для 
$A_{i, j} = A_{i, j} - A_{i, k} * A_{k, j}, \quad i = 1,...,k-1,k+1,...,q,
\quad j = k + 1,..,q$:\\
$$\sum\limits_{k=1}^{q-1}(q-k)*2*m^{3}*(q-1)$$
Тогда, суммируя, получим: $n^{3} + n*m^{2} + O(n^{2} + m*m + m^{2})$\\
Для присоеденнёной матрицы:\\
Сложность умножений на блок, обратный к ведущему $B_{k, j} = A_{k, k}^{-1} * B_{k, j},
\quad j = 1,...,k-1$ :\\
$$\sum\limits_{k=1}^{q}(k-1)*(2*m^{3} - m^{2})$$
Сложность умножений и вычитаний для 
$B_{i, j} = B_{i, j} - A_{i, k} * B_{k, j}, \quad i = 1,...,k-1,k+1,...,q,
\quad j = 1,...,k$:\\
$$(q-1)*\sum\limits_{k=1}^{q}*k*2*m^{3}$$
Тогда, суммируя, получим: $n^{2}*m + n^{3} - m^{2} * n + O(n^{2} + m*m + m^{2})$\\
Итог: $$2*n^{3} + n^{2}*m - n*m^{2} + O(n^{2} + m*m + m^{2})$$

\section{Проверка формулы (вариант 2)}
$S(n, 1) = 2*n^{3}$ - сложность обычного матода Жордана нахождений обратной матрицы\\
$S(n, n) = 2*n^{3}$ - сложность обычного матода Жордана нахождения обратной матрицы,
но нет ешё одного умнодения, так как в присоеденённой матрице еденичный блок на 
$B_{k, k}$ на шаге $k$ на $A_{k,k}^{-1}$ не умножается, а просто заменяется
(если все таки умножать, то получится вариант 1, то есть при $m=n$ будет
сложность = Жордан + одно умножений = $4*n^{3})$

\end{document}










